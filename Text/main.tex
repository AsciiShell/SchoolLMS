%!TEX TS-program = xelatex

% Шаблон документа LaTeX создан в 2018 году
% Алексеем Подчезерцевым
% В качестве исходных использованы шаблоны
% 	Данилом Фёдоровых (danil@fedorovykh.ru) 
%		https://www.writelatex.com/coursera/latex/5.2.2
%	LaTeX-шаблон для русской кандидатской диссертации и её автореферата.
%		https://github.com/AndreyAkinshin/Russian-Phd-LaTeX-Dissertation-Template

\documentclass[a4paper,14pt]{article}

\input{data/preambular.tex}
\begin{document} % конец преамбулы, начало документа
\begin{titlepage}
	\begin{center}
		ФЕДЕРАЛЬНОЕ  ГОСУДАРСТВЕННОЕ АВТОНОМНОЕ \\
		ОБРАЗОВАТЕЛЬНОЕ УЧРЕЖДЕНИЕ ВЫСШЕГО ОБРАЗОВАНИЯ\\
		«НАЦИОНАЛЬНЫЙ ИССЛЕДОВАТЕЛЬСКИЙ УНИВЕРСИТЕТ\\
		«ВЫСШАЯ ШКОЛА ЭКОНОМИКИ»
	\end{center}
	
	\begin{center}
		\textbf{Московский институт электроники и математики}
		
		\textbf{Им. А.Н.Тихонова НИУ ВШЭ}
	\end{center}
	\vspace{1ex}	
	\begin{center}
		Солодянкин Андрей Александрович, группа БИВ172\\
		Подчезерцев Алексей Евгеньевич, группа БИВ172
	\end{center}	
	\vspace{1ex}
	\begin{center}
		\textbf{ТЕХНИЧЕСКОЕ ЗАДАНИЕ\\
		ПО МЕЖДИСЦИПЛИНАРНОЙ КУРСОВОЙ РАБОТЕ
	}
	\end{center}	
	\vspace{2ex}
	\begin{center}
		по теме «Проектирование и разработка электронной системы для дополнительного образования со школьниками»\\
	\end{center}
	\vspace{2ex}
	\begin{center}
%	Дата сдачи отчета: \today
	\end{center}
	\vspace{2ex}
	\vfill
	\begin{center}
		Москва \the\year г.
	\end{center}
\end{titlepage}

\section*{Аннотация}

%Количество страниц работы: \getlastpage, иллюстраций: \totalfigures, таблиц: \totaltables, источников: \LastBib.

\pagebreak

\section*{Annotation}

%Total number of pages is \getlastpage, figures: \totalfigures, tables: \totaltables, cite sources: \LastBib.
\pagebreak

\tableofcontents
\pagebreak

\section*{Введение}

% Начало введения

% Актуальность

% Новизна

% Объект

% Предмет

% Цель

% Задачи

% Значимость

% Личный вклад

% Конец введения

\pagebreak

% ==============================================================================
% Начало Г1

\section{Обзор и анализ предметной области}

% Comment - надо аккуратно перенести часть текста во 2 главу, чтобы GIT не ругался

\subsection{Состояние предметной области}

Важной частью современного образования является дополнительные образовательные программы, которые направлены на разностороннее развитие личных качеств.
Секции, кружки по интересам, образовательные курсы получили широкое распространение во многих странах мира.
Некоторые учебные программы финансируются государством, другие привлекают школьников или студентов для дальнейшей учёбы или работы.

Современный курс может носить как дистанционный, так и очный характер, однако в любом случае необходимо организовать данный процесс.
Преподавателям и руководителям требуется контролировать запись и посещаемость, распространение справочного материала, проводить срез знаний, проверять работы, вести учёт оценок и выполнять множество других однотипных действий.
Логичным решением данной проблемы есть автоматизация данного процесса с использованием программных средств.

Данный подход широко используется в современном образовании. В частности, многие организации используют собственные системы контроля образовательного процесса или подключаются к уже готовым решениям.

% Г1 - детальное и подробное описание существующих технологий

% Г1 - детальное и подробное описание аналогичных технологических решений

% Г1 - анализ технологий

% Г1 - Выбор Решения - таблица сравнения

% Г1 - Выводы и заключение

% Конец Г1
% ==============================================================================
% Начало Г2

% Методы решения

\section{Обзор и анализ методов решения}

\subsection{Анализ существующих решений}

%Существует огромное количество систем управления обучением, 

%В последнее время появилось огромное количество систем для управления образованием. 
Для сравнения были выбраны следующие варианты:

\begin{itemize}
	\item iSpring Online LMS
	\item Moodle	
	\item edx
	\item LMS HSE
	
\end{itemize}

% Детальное описание

\subsubsection{iSpring Online LMS}

Аналогов iSpring Online LMS существует очень много.
%Их объединяет следующие преимущества этих систем: простота в установке и управлении
К основными чертам таких систем можно отнести:
\begin{itemize}
	\item простота в установке
	\item простота в управлении	
	\item профессиональная техническая поддержка.
\end{itemize}

При этом у них присутствуют следующие недостатки:
\begin{itemize}
	\item отсутствие кастомизации
	\item стоимость системы
\end{itemize} 

\subsubsection{LMS eFront }

\begin{itemize}
	\item Чёткая вёрстка веб-страниц, стабильная работа программной оболочки 
	\item Подробные отчёты о деятельности пользователей с гибкой фильтрацией	
	
\end{itemize} 

Недостатки системы:

\begin{itemize}
	\item Относительно небольшое количество инструментов для создания учебных материалов
	\item Относительно мало дополнительных модулей
	\item Малое сообщество пользователей
	
	
\end{itemize} 

\subsubsection{Moodle}

Moodle - крупнейшая и самая популярная система управления курсами, которая активно развивается с 2002 года. Важными преимуществами этой системы являются:

\begin{itemize}
	\item бесплатность системы
	\item Неограненный возможности кастомизации
	\item Наличие широкого функционала для обеспечения процесса обучения (есть почти всё)
	\item Бесплатная, с открытым исходным кодом
	\item Большое онлайн-сообщество на официальном сайте
	
\end{itemize} 

Минусы системы:

\begin{itemize}
	\item Сложна для освоения	
\end{itemize} 


\subsubsection{Open edX} 

Платформа Open edX была создана Массачусетским технологическим институтом совместно с Гарвардским университетом.
Данная платформа позволяет создавать массовые онлайн курсы, ориентированные для широкой аудитории.

К достоинствам данного решения можно отнести:

\begin{itemize}
	\item Открытый исходный код платформы
	\item Разнообразие плагинов для расширения функционала
	\item Большое сообщество разработчиков
\end{itemize}

Недостатки:

\begin{itemize}
	\item Система ориентирована на дистанционное обучение
	\item Сложность с интеграцией с существующими проектами образовательного учреждения
\end{itemize}

\subsubsection{Собственное решение} 

\textbf{Написать решение с <<Нуля>>} часто рассматривается при проектировании.
Оно позволит создать продукт с функционалом, наиболее подходящим к желанию заказчика.

Достоинства подхода:

\begin{itemize}
	\item Возможность создать продукт с максимальным соответствием требованиям
	\item Отсутствие зависимостей от сторонних готовых CMS 
\end{itemize}

Однако, данный подход содержит существенный недостаток: высокая стоимость разработки (в человеко-часах).
Разработчикам придётся создать весь функционал, который существует в готовых решениях,  и ещё добавить связанный с конкретной задачей.
Кроме того, не будет достаточно времени и ресурсов для полного тестирования системы.

% Анализ методов

% TODO совпадение с предыдущим пунктом?

% Выбор метода

\begin{landscape}
	\begin{table}[!h]
		\begin{center}
			\begin{flushleft}
				\tablecaption{Сравнение основных систем}
			\end{flushleft}
			
			\begin{tabular}{|l|l|l|l|l|l|}
				\hline
				& iSpring Online LMS & LMS eFront & Moodle & Open edX & Собственное решение \\ \hline
				Бесплатно                              & -                  & +          & +      & +        & +                   \\ \hline
				Малое время разработки                 & +                  & +          & +      & +        & -                   \\ \hline
				Наличие широкого сообщества            & техподдержка       & ±          & +      & +        & -                   \\ \hline
				Возможность кастомизации               & -                  & ±          & ±      & ±        & +                   \\ \hline
				Простота разработки                    & -                  & ±          & +      & +        & -                   \\ \hline
				Простота обслуживания                  & +                  & ±          & ±      & ±        & +                   \\ \hline
				Надежность                             & +                  & +          & +      & +        & -                   \\ \hline
				Ориентация на традиционное образование & +                  & +          & +      & -        & +                   \\ \hline
			\end{tabular}
		\end{center}
	\end{table}
\end{landscape}

% Заключение и выводы

% Конец Г2
% ==============================================================================
% Начало Г3
\section{Теоретические основы системы} 

Основное отличие нашего решения от всевозможных модернизаций Moodle -- наличие интеграции с серверами НИУ ВШЭ для автоматического управления расписанием.
Необходимо разработать структуру хранения, получения, обработки и вывода информации о занятиях.

\subsection{Проектирование баз данных}

Новая система должна хранить расписание занятий для ускорения обработки данных.
Кроме того, каждый курс должен знать информацию о номере группы в РУЗе для получения обновлений.
На первый взгляд, логичным решением будет добавление дополнительного атрибута в таблицу курса, который будет указывать на номер группы, однако такое решение может повлиять на внутреннюю работу системы Moodle, поэтому было решено создать дополнительную таблицу со связью 1 к 1 с таблицей курса.
Дополнительно была создана таблица для хранения кэшей расписание, содержащая основные атрибуты, возвращаемые сервисом расписания, и номер группы, к которой относится конкретное занятий (Рис. \ref{img:db_struct}).

\begin{figure}[H]
	\centering		
	\includegraphics[width=\linewidth]{schemas/database}
	\caption{Схема базы данных}\label{img:db_struct}
\end{figure}

\subsection{Проектирование классов и методов}

Для удобства обращения к другим WEB ресурсам была создана обёртка над стандартной программой cURL, предназначенной для отправки запросов на другие сервера (Рис. \ref{img:request_class}).
Итоговый вариант в конструкторе устанавливал параметры для отправки запроса и, в случае успеха ответа, приводил ответ из JSON формата в структуры PHP.
После окончания работы с запросом соединение автоматически закрывалось.

\begin{figure}[H]
	\centering		
	\includegraphics{image/RequestsGet}
	\caption{Структура класса RequestsGet}\label{img:request_class}
\end{figure}

Более важным в проекте является класс для работы с базой данных курса (Рис. \ref{img:db_class}). 
Данная структура агрегирует различные методы для создания, изменения и удаления сущностей.
Объект был создан с применением паттерна Singleton, так как сам класс не хранит данные, а лишь обращается к базе данных.
При первом создании экземпляра проверяется наличие необходимых таблиц, в случае их отсутствия они будут автоматически созданы.
Простые запросы реализованы с помощью Moodle ORM для сокращения кода, более сложные со множеством условий и соединением таблиц написаны вручную.


\begin{figure}[H]
	\centering		
	\includegraphics[width=0.7\linewidth]{image/DataBaseClass}
	\caption{Структура класса DataBaseCourse}\label{img:db_class}
\end{figure}

\subsection{Проектирование пользовательского интерфейса}

После установки и знакомством с пользовательским интерфесом администратора были обнаружены его недостатки: перегруженность и сложность для нового пользователя.
Переходы между разделами бывают не совсем логичными или доступны при установке дополнительных параметров на специальных страницах.
Было решено спроектировать отдельную страницу для управления администратором связями с данными расписания и выполнением базовых действий с существующими, новыми и будущими пользователями.
Логика переходов отражена на Рис. \ref{img:UI}.

\begin{figure}[H]
	\centering		
	\includegraphics[width=\linewidth]{schemas/UI}
	\caption{Схема переходов в пользовательском интерфейсе}\label{img:UI}
\end{figure}

В центре синим цветом изображены страницы, непосредственно с которых происходит взаимодействия пользователя.
Стрелками показаны возможные переходы между страницами и действия на них.
Красными выделены блоки, которые являются частью интерфейса Moodle.

% Модели

% Методы

% Алгоритмы

% Схемы

% UML

% Выводы

% Конец Г3
% ==============================================================================
% Начало Г4

% Описание решения

% Описание экспериментов

% Демонстрация

% Анализ результатов

% Подтверждение актуальности

% Выводы и заключение

% Конец Г4
% ==============================================================================
% Начало заключения

% Результаты

% Научная новизна и практическая значимость

% Личный вклад каждого соавтора работы, даже если автор 1

% Ваши выводы

% Предполагаемое применение полученных результатов

% Направление дальнейших исследований (перспективы дальнейшего развития работы)

% Конец заключения
% ==============================================================================
\end{document} % конец документа

